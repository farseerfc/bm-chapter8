%\documentclass[CJK]{beamer}  

\usepackage{fontspec,xunicode,xltxtra,beamerthemesplit,listings}

\usepackage{tikz}
\usetikzlibrary{shapes,arrows}

\usetheme{Luebeck} 
\useoutertheme{infolines}

\setmainfont{MS PGothic}
\setsansfont{Meiryo UI} %[Mapping=tex-text]
%\setmonofont{Bitstream Vera Sans Mono}
\setmonofont{Courier New}


\def\beamer@linkspace#1{%
  \begin{pgfpicture}{0pt}{-1.5pt}{#1}{5.5pt}
    \pgfsetfillopacity{0}
    \pgftext[x=0pt,y=-1.5pt]{.}
    \pgftext[x=#1,y=5.5pt]{.}
  \end{pgfpicture}}

\lstset{tabsize=4, %
  frame=shadowbox, %把代码用带有阴影的框圈起来
  commentstyle=\color{red!50!green!50!blue!50},%浅灰色的注释
  rulesepcolor=\color{red!20!green!20!blue!20},%代码块边框为淡青色
  keywordstyle=\color{blue!90}\bfseries, %代码关键字的颜色为蓝色,粗体
  showstringspaces=false,%不显示代码字符串中间的空格标记
  stringstyle=\ttfamily, % 代码字符串的特殊格式
  keepspaces=true, %
  breakindent=22pt, %
  numbers=left,%左侧显示行号
  stepnumber=1,%
  numberstyle=\tiny, %行号字体用小号
  basicstyle=\footnotesize, %
  showspaces=false, %
  flexiblecolumns=true, %
  breaklines=true, %对过长的代码自动换行
  breakautoindent=true,%
  breakindent=4em, %
  aboveskip=1em, %代码块边框
  %% added by http://bbs.ctex.org/viewthread.php?tid=53451
  fontadjust,
  captionpos=t,
  framextopmargin=2pt,framexbottommargin=2pt,abovecaptionskip=-3pt,belowcaptionskip=3pt,
  xleftmargin=4em,xrightmargin=4em, % 设定listing左右的空白
  texcl=true,
  % 设定中文冲突,断行,列模式,数学环境输入,listing数字的样式
  extendedchars=false,columns=flexible,mathescape=true
  numbersep=-1em
}


\setbeamercovered{transparent}

%%%%%%%%%%%%%%%%%%%%%%%%%%%%%%%%%%%%%%%%%%%%%%%%%%%%%%%%%%%%%%%%%%%%%%%%

\title[チェックリストと区画に基づく網羅率と使用テスト]{
チェックリストと区画に基づく\\
網羅率と使用テスト\\
{\small\scshape Coverage and Usage Testing
    \\Based on Checklists and Partitions}}

\subtitle[B4M1 輪講]{第8章 ($p107\sim p126$) B4M1 輪講}

\author[大阪大学大学院CS専攻\quad{}楊 嘉晨]{修士課程1年生\quad{}楊 嘉晨}
\institute[楠本研]{大阪大学大学院 コンピュータサイエンス専攻 楠本研究室}
\date{2012年5月29日(火)}

%%%%%%%%%%%%%%%%%%%%%%%%%%%%%%%%%%%%%%%%%%%%%%%%%%%%%%%%%%%%%%%%%%%%%%%%
\begin{document} 

\newcommand{\toprule}{\hline}
\newcommand{\midrule}{\hline}
\newcommand{\bottomrule}{\hline}
\newcommand{\itemtitle}[1]{\item \alert{#1} \quad{} }
\newcommand{\upcite}[1]{ }

\XeTeXlinebreaklocale "jp"  
\XeTeXlinebreakskip = 0pt plus 1pt 

\frame{\titlepage} 

\begin{frame}{目次}
\tableofcontents 
\end{frame}


\AtBeginSubsection[]{
  \begin{frame}{目次}
    \tableofcontents[currentsection,currentsubsection] 
  \end{frame}
}

%%%%%%%%%%%%%%%%%%%%%%%%%%%%%%%%%%%%%%%%%%%%%%%%%%%%%%%%%%%%%%%%%%%%%%%%
\section{第8章の概要}
\subsection{概要}
\begin{frame}{概要}

この章に、配列や区画とか簡単なモデルで正規テストの手法について紹介します。

\begin{enumerate}
\item 8.1節に、いろんなチェックリストで正規と半正規のテスト
\item 8.2節に、チェックリストを区画に正規化して、簡単な網羅率テストを行い
\item 8.3節に、操作履歴(Operation Profile, OP)という、区画のために簡単な使用ベースのテスト
\item 8.4節に、OPを更に発展する手順
\item 8.5節に、事例研究
\end{enumerate}

第9章には、区画した入力領域の境界テストの条件について、似ているモデルを紹介します。

\end{frame}
%%%%%%%%%%%%%%%%%%%%%%%%%%%%%%%%%%%%%%%%%%%%%%%%%%%%%%%%%%%%%%%%%%%%%%%%

\section{8.1 チェックリストに基づくテスト、とその制限}
\subsection{チェックリストに基づくテスト}
\begin{frame}{Ad hotテスト \& ランダムテスト}
\pgfdeclareimage[width=0.8\textwidth]{Ad-Hoc-Testing}{figure/Ad-Hoc-Testing.jpg}
\pgfdeclareimage[width=\textwidth]{Random-Testing}{figure/Random-Testing.jpg}

\begin{columns}[t]
\begin{column}{.5\textwidth}
\begin{figure}
\pgfuseimage{Ad-Hoc-Testing}
\caption{Ad hot testing}
\end{figure}
\end{column}
\begin{column}{.5\textwidth}
\begin{figure}
\pgfuseimage{Random-Testing}
\caption{Random testing}
\end{figure}
\end{column}
\end{columns}

\end{frame}

%%%%%%%%%%%%%%%%%%%%%%%%%%%%%%%%%%%%%%%%%%%%%%%%%%%%%%%%%%%%%%%%%%%%%%%%

\begin{frame}{チェックリストに基づくテスト}
\end{frame}

\end{document}
