%\documentclass[CJK]{beamer}  


\usepackage{fontspec,xunicode,xltxtra,beamerthemesplit,listings}

\usepackage{tikz}
\usetikzlibrary{shapes,arrows}

\usetheme{Warsaw} 
\useoutertheme{infolines}

\setbeamercovered{transparent}

\renewcommand{\figurename}{図}
\newtheorem{definationfc}{定義}

\usepackage[overlap,CJK]{ruby}
\usepackage{multicol}
\usepackage[sort&compress]{natbib}


\setmainfont{MS PGothic}
\setsansfont{Meiryo UI} %[Mapping=tex-text]
\setmonofont{Courier New}


\def\beamer@linkspace#1{%
  \begin{pgfpicture}{0pt}{-1.5pt}{#1}{5.5pt}
    \pgfsetfillopacity{0}
    \pgftext[x=0pt,y=-1.5pt]{.}
    \pgftext[x=#1,y=5.5pt]{.}
  \end{pgfpicture}}

\lstset{tabsize=4, %
  frame=shadowbox, %把代码用带有阴影的框圈起来
  commentstyle=\color{red!50!green!50!blue!50},%浅灰色的注释
  rulesepcolor=\color{red!20!green!20!blue!20},%代码块边框为淡青色
  keywordstyle=\color{blue!90}\bfseries, %代码关键字的颜色为蓝色,粗体
  showstringspaces=false,%不显示代码字符串中间的空格标记
  stringstyle=\ttfamily, % 代码字符串的特殊格式
  keepspaces=true, %
  breakindent=22pt, %
  numbers=left,%左侧显示行号
  stepnumber=1,%
  numberstyle=\tiny, %行号字体用小号
  basicstyle=\footnotesize, %
  showspaces=false, %
  flexiblecolumns=true, %
  breaklines=true, %对过长的代码自动换行
  breakautoindent=true,%
  breakindent=4em, %
  aboveskip=1em, %代码块边框
  %% added by http://bbs.ctex.org/viewthread.php?tid=53451
  fontadjust,
  captionpos=t,
  framextopmargin=2pt,framexbottommargin=2pt,abovecaptionskip=-3pt,belowcaptionskip=3pt,
  xleftmargin=4em,xrightmargin=4em, % 设定listing左右的空白
  texcl=true,
  % 设定中文冲突,断行,列模式,数学环境输入,listing数字的样式
  extendedchars=false,columns=flexible,mathescape=true
  numbersep=-1em
}

%%%%%%%%%%%%%%%%%%%%%%%%%%%%%%%%%%%%%%%%%%%%%%%%%%%%%%%%%%%%%%%%%%%%%%%%

\title[チェックリストと分割に基づく網羅と使用テスト]{
チェックリストと分割に基づく 網羅と使用テスト\\
{\small\scshape Coverage and Usage Testing Based on Checklists and Partitions}}

\subtitle[B4M1 輪講]{第8章 ($p107\sim p126$) B4M1 輪講}

\author[大阪大学大学院CS専攻\quad{}楊 嘉晨]{修士課程1年生\quad{}楊 嘉晨}
\institute[楠本研]{大阪大学大学院 コンピュータサイエンス専攻 楠本研究室}
\date{2012年5月29日(火)}

\pgfdeclareimage[height=0.618cm]{logo}{figure/logo.pdf}
\logo{\pgfuseimage{logo}}

\mode<article>{\providecommand{\imageheight}{0.2\textheight}}
\mode<presentation>{\providecommand{\imageheight}{0.4\textheight}}

%%%%%%%%%%%%%%%%%%%%%%%%%%%%%%%%%%%%%%%%%%%%%%%%%%%%%%%%%%%%%%%%%%%%%%%%
\begin{document} 

\mode<presentation>{

\providecommand{\newblock}{\\}
\providecommand{\toprule}{\hline}
\providecommand{\midrule}{\hline}
\providecommand{\bottomrule}{\hline}
\providecommand{\itemtitle}[1]{\item \alert{#1} \quad{} }
\providecommand{\upcite}[1]{ }
}

\XeTeXlinebreaklocale "jp"  
\XeTeXlinebreakskip = 0pt plus 1pt 

\frame{\titlepage} 
\mode<article>{\maketitle}

\begin{frame}<trans|beamer>[shrink=10]{目次}
\tableofcontents[section,subsectionstyle=hide,subsubsectionstyle=hide] 
\end{frame}


\AtBeginSubsection[]{
  \begin{frame}<trans|beamer>[shrink=25]{小節目次}
    \tableofcontents[sectionstyle=show/shaded,subsectionstyle=show/shaded/hide] 
  \end{frame}
}

%%%%%%%%%%%%%%%%%%%%%%%%%%%%%%%%%%%%%%%%%%%%%%%%%%%%%%%%%%%%%%%%%%%%%%%%
\section{第8章の概要}
\subsection{概要}
\begin{frame}{概要}

%\transwipe[direction=-90]

配列や分割(Partition)とか簡単なモデルで正規テストの手法について紹介します

\begin{enumerate}
\item 8.1節に、いろんなチェックリストで正規と半正規のテスト
\item 8.2節に、チェックリストを分割に正規化して、簡単な網羅テストを行い \note[item]{網羅:もうら}
\item 8.3節に、操作分布(Operation Profile, OP)という、分割のために簡単な使用ベースのテスト
\item 8.4節に、OPを更に発展する手順
\item 8.5節に、事例研究 \note[item]{事例研究:じれいけんきゅう}
\end{enumerate}

第9章には、分割した入力サブドメインの境界条件のテストについて、似ているモデルを紹介します。

\end{frame}
%%%%%%%%%%%%%%%%%%%%%%%%%%%%%%%%%%%%%%%%%%%%%%%%%%%%%%%%%%%%%%%%%%%%%%%%

\section{8.1 チェックリストに基づくテスト、とその制限}
\subsection{チェックリストに基づくテスト(Testing with Checklists)}

\begin{frame}{Ad hocテスト \& ランダムテスト}
\framesubtitle{Ad hoc Testing \& Random Testing}


\pgfdeclareimage[height=\imageheight]{Ad-Hoc-Testing}{figure/Ad-Hoc-Testing.jpg}
\pgfdeclareimage[height=\imageheight]{Random-Testing}{figure/Random-Testing.jpg}

\transwipe[direction=0]<2>
\transwipe[direction=180]<3>

\mode<presentation>{
\begin{columns}[t]
\begin{column}{.5\textwidth}
\only<1,3>{
\begin{figure}
\begin{center}
\pgfuseimage{Ad-Hoc-Testing}
\caption{Ad hot Testing}
\end{center}
\end{figure}

}

\end{column}
\begin{column}{.5\textwidth}
\only<2,3>{
\begin{figure}
\begin{center}
\pgfuseimage{Random-Testing}
\caption{Random Testing}
\end{center}
\end{figure}
}
\end{column}
\end{columns}

\begin{overprint}
\onslide<1|handout:0|trans:0> ソフトウェアを\alert{勝手に}実行してその結果を観察
\onslide<2|handout:0|trans:0> \alert{統計的}にランダムの分布でテストの操作を行い
\onslide<3|handout:1|trans:1> 繰り返してAd hocテストを実行する時、試験者は行ったテストを追跡すると、チェックリストテストになった
\end{overprint}
}

\mode<article>{
図~\ref{fig:adhoc}ソフトウェアを\alert{勝手に}実行してその結果を観察

図~\ref{fig:random} \alert{統計的}にランダムの分布でテストの操作を行い

繰り返してAd hocテストを実行する時、試験者は行ったテストを追跡すると、チェックリストテストになった

\begin{figure}
\begin{center}
\label{fig:adhoc}
\pgfuseimage{Ad-Hoc-Testing}
\caption{Ad Hot Testing}
\end{center}
\end{figure}
\begin{figure}
\begin{center}
\label{fig:random}
\pgfuseimage{Random-Testing}
\caption{Random Testing}
\end{center}
\end{figure}
}

\end{frame}

%%%%%%%%%%%%%%%%%%%%%%%%%%%%%%%%%%%%%%%%%%%%%%%%%%%%%%%%%%%%%%%%%%%%%%%%

\begin{frame}[shrink=15]{チェックリストに基づくテスト}
\framesubtitle{Testing with checklists}
\transfade<2>[duration=0.5]
\mode<article>{\renewcommand{\imageheight}{0.2\textheight}}
\mode<presentation>{\renewcommand{\imageheight}{0.5\textheight}}
\pgfdeclareimage[height=\imageheight]{checklist}{figure/checklist.jpg}
\pgfdeclareimage[height=\imageheight]{blackboxchecklist}{figure/blackboxchecklist.png}
\begin{columns}[t]
    \begin{column}{.6\textwidth}
    \begin{itemize}
    \item 黒箱テスト
        \begin{itemize}
        \item ソフトウェア要求チェックリスト
        \item 機能チェックリスト
            \begin{itemize}
            \item システム全体から高レベルの機能
            \item 低レベルの字句、独立な部品
            \end{itemize}
        \end{itemize}
    \item 白箱テスト
        \begin{itemize}
        \item プログラムの特徴
        \item コーディング標準
        \item 単体テストの文の網羅
        \item 統合テストとシステムテストの部品の網羅
        \item 構造と特徴など、実装に関しるチェックリスト
            \begin{itemize}
            \item 関数の呼び出し規則
            \item 資源の生産者と消費者
            \item モジュール間に共有するデータ
            \end{itemize}
        \end{itemize}
    \end{itemize}
    \end{column}
        \begin{column}{.35\textwidth}
        \begin{figure}
        \begin{center}
            \begin{overlayarea}{\textwidth}{0.6\textheight}
            \begin{center}
            \only<1|handout:0|trans:0|article:0>{\pgfuseimage{checklist}}
            \only<2|handout:1|trans:1>{\pgfuseimage{blackboxchecklist}}
            \end{center}
            \end{overlayarea}
        \caption{チェックリスト}
        \end{center}
        \end{figure}
    \end{column}
\end{columns}
\end{frame}
%%%%%%%%%%%%%%%%%%%%%%%%%%%%%%%%%%%%%%%%%%%%%%%%%%%%%%%%%%%%%%%%%%%%%%%%
\subsection{階層及び複合チェックリスト(Hierarchical and Combined Checklists)}
\begin{frame}{階層チェックリスト}
\framesubtitle{Hierarchical Checklists }
\note[item]{階層:かいそう}
\begin{columns}
    \begin{column}{0.35\textwidth}
    \flushright{
    \begin{enumerate}
    \item 高レベル項目1
    \item 高レベル項目2
    \item 高レベル項目3
    \end{enumerate}
    }
    \end{column}
    \begin{column}{0.05\textwidth}
    \Huge\[\Rightarrow\]
    \end{column}
    \begin{column}{0.5\textwidth}
    \begin{enumerate}
    \item 高レベル項目1
        \begin{enumerate}
        \item 低レベル項目1
        \item 低レベル項目2
        \item 低レベル項目3
        \end{enumerate}
    \item 高レベル項目2
        \begin{enumerate}
        \item 低レベル項目4
        \item 低レベル項目5
        \end{enumerate}
    \item 高レベル項目3
        \begin{enumerate}
        \item 低レベル項目6
        \item 低レベル項目7
        \item 低レベル項目8
        \end{enumerate}
    \end{enumerate}
    \end{column}
\end{columns}
\end{frame}
%%%%%%%%%%%%%%%%%%%%%%%%%%%%%%%%%%%%%%%%%%%%%%%%%%%%%%%%%%%%%%%%%%%%%%%%
\begin{frame}[shrink=5]{複合チェックリスト}
\framesubtitle{Combined Checklists}

\note[item]{複合:ふくごう}
\pgfdeclareimage[height=\imageheight]{2dchecklist}{figure/2dchecklist.png}
\transfade<2->[duration=0.5]

\begin{tabular}{lc}
& \begin{actionenv}<uncover@3|alert@3|article:0> 要求1 \quad 要求2 \quad 要求3\end{actionenv} \\
\begin{minipage}{0.3\textwidth}
\begin{enumerate}
\begin{actionenv}<uncover@1-3>
\item 部品1
    \begin{actionenv}<uncover@2,3 | alert@3>
    \begin{enumerate}
    \item 要求1
    \item 要求2
    \item 要求3
    \end{enumerate}
    \end{actionenv}
\item 部品2
    \begin{actionenv}<uncover@2,3 | alert@3>
    \begin{enumerate}
    \item 要求1
    \item 要求2
    \item 要求3
    \end{enumerate}
    \end{actionenv}
\item 部品3
    \begin{actionenv}<uncover@2,3 | alert@3>
    \begin{enumerate}
    \item 要求1
    \item 要求2
    \item 要求3
    \end{enumerate}
    \end{actionenv}
\end{actionenv}
\end{enumerate} 
\end{minipage} & 
\mode<presentation>{\visible<4>{
    \begin{minipage}{0.4\textwidth}
        \begin{figure}
            \begin{center}
            \pgfuseimage{2dchecklist}
            \end{center}
        \caption{標準チェックリストと部品チェックリストを複合した二次元表}
        \end{figure}
    \end{minipage}
}}
\end{tabular}

\mode<article>{
\begin{figure}
    \begin{center}
    \pgfuseimage{2dchecklist}
    \end{center}
\caption{標準チェックリストと部品チェックリストを複合した二次元表}
\end{figure}
}
\end{frame}

%%%%%%%%%%%%%%%%%%%%%%%%%%%%%%%%%%%%%%%%%%%%%%%%%%%%%%%%%%%%%%%%%%%%%%%%
\subsection{チェックリストの問題点と制限(Problems and Limits of Checklists)}
\begin{frame}{チェックリストの問題点}
\framesubtitle{Problems General Checklists}

\begin{itemize}
\item チェックリストの定義が抽象過ぎて、具体的にするのが難しい
    \begin{itemize}
    \item 特に高レベル
    \end{itemize}
\item 具体的なテストケースに変換するのは
    \begin{itemize}
    \item 経験が必要
    \item 特殊な環境や設定等に依存
    \end{itemize}
\item 大規模な、複雑なシステムに、相互接続と相互作用を定義するのも難しい 
    \note[item]{相互接続:そうごせつぞく}
    \note[item]{相互作用:そうごさよう}
\end{itemize}

\end{frame}
%%%%%%%%%%%%%%%%%%%%%%%%%%%%%%%%%%%%%%%%%%%%%%%%%%%%%%%%%%%%%%%%%%%%%%%%
\begin{frame}{チェックリストの制限}
\framesubtitle{Limitations of General Checklists}
\transfade<2->[duration=0.5]
\begin{enumerate}
\item 全部の機能(黒箱テスト)又は構造部品(白箱テスト)を、異なる視点や保証レベルから、網羅することが難しい
    \begin{itemize}
    \item 網羅されていない穴(Hole)が残ってしまい
    \item<uncover@3-|alert@3> 分割したチェックリストを使う
    \end{itemize}
\item より高い網羅率を着目すると、重複したテストを取り込んでしまい
    \begin{itemize}
    \item 無駄なテストを行われてしまい
    \item<uncover@3-|alert@3> 分割したチェックリストを使う
    \end{itemize}
\item 各システムの部品間の複雑な相互作用を定義するのは難しい
    \begin{itemize}
    \item<uncover@2-|alert@2> 10、11章にFSM基づく体系的な正規モデルを紹介
    \note[item]{体系的:たいけいてき} \note[item]{FSM: 有限状態機械(ゆうげんじょうたいきかい)、有限オートマトン}
    \end{itemize}
\end{enumerate}
\end{frame}
%%%%%%%%%%%%%%%%%%%%%%%%%%%%%%%%%%%%%%%%%%%%%%%%%%%%%%%%%%%%%%%%%%%%%%%%
\section{8.2 分割網羅テスト(Testing for Partition Coverage)}
\subsection{紹介}
\begin{frame}{分割網羅テスト}
\framesubtitle{Testing for Partition Coverage}
\begin{itemize}
\item 分割と分割を基づくテストは一種のチェックリスト・テスト
\item 分割は集合全体を徹底的に覆う \note[item]{覆う:おおう} \note[item]{徹底的:てっていてき}
    \begin{itemize}
    \item より高い\alert{網羅率}
    \end{itemize}
\item 分割はお互いに重複ことができない
    \begin{itemize}
    \item より高い\alert{効率}
    \end{itemize}
\end{itemize}
\end{frame}
%%%%%%%%%%%%%%%%%%%%%%%%%%%%%%%%%%%%%%%%%%%%%%%%%%%%%%%%%%%%%%%%%%%%%%%%
\subsection{8.2.1 動機の事例(Some Motivational Examples)}
\begin{frame}{動機の事例}
\framesubtitle{Some Motivational Examples}
\mode<article>{\renewcommand{\imageheight}{0.15\textheight}}
\mode<presentation>{\renewcommand{\imageheight}{0.35\textheight}}
\transfade<2->[duration=0.5]
\begin{columns}
\begin{column}{0.5\textwidth}
\[ ax^2+bx+c = 0 \]
\pause
その解を求めると
\[ r=\frac{-b \pm \sqrt{b^2-4ac}}{2a} \]
\pause
全部可能な入力の組み合わせ
\[ 2^{32} \times 2^{32} \times 2^{32} = 2^{96}\]
\end{column}
\begin{column}{0.5\textwidth}
\only<4->{
\pgfdeclareimage[height=\imageheight]{axbxc}{figure/axbxc.png}
\begin{figure}
\begin{center}
\pgfuseimage{axbxc}
\caption{$ax^2+bx+c=0$のテストケース}
\end{center}
\end{figure}
}
\end{column}
\end{columns}
\end{frame}
%%%%%%%%%%%%%%%%%%%%%%%%%%%%%%%%%%%%%%%%%%%%%%%%%%%%%%%%%%%%%%%%%%%%%%%%
\subsection{8.2.2 分割:概念と定義(Partition: Concepts and Definitions)}
\begin{frame}[shrink=10]{分割:概念と定義}
\framesubtitle{Partition: Concepts and Definitions}
\begin{columns}
\begin{column}{0.5\textwidth}
\begin{enumerate}
\item 分割した集合は相互に排他し、一つの要素が2つの集合に共有されていることがない
    \[ \forall i,j,i\neq j \Rightarrow G_i \cap G_j = \emptyset \]
\item 分割した集合の和集合は全体の集合、和集合に分割する前に元の集合の全てが含まれている
    \[ \bigcup_{i=1}^{n} G_i = S \]
\end{enumerate}

\end{column}
\begin{column}{0.5\textwidth}
分割した集合は同値類になる。

同値関係を持つには
\begin{enumerate}
\item 反射率 \\ $R(a,a)$ が常に成り立つ
\item 対称律 \\ $R(a,b) \Rightarrow R(b,a) $
\item 推移律 \\ $R(a,b) \wedge R(b,c) \Rightarrow R(a,c)$
\end{enumerate}
\end{column}
\end{columns}
\end{frame}
%%%%%%%%%%%%%%%%%%%%%%%%%%%%%%%%%%%%%%%%%%%%%%%%%%%%%%%%%%%%%%%%%%%%%%%%
\subsection{8.2.3 分割テストの決定と網羅範囲の予測}
\begin{frame}{分割テストの網羅範囲の決定}
\framesubtitle{Testing Decisions for Partition Coverage}
分割テストは一種のチェックリストであり、分割の種類は8.1節に述べたチェックリストの種類に似ている。但し、分割の決定は以下のとおりに決められている
\begin{enumerate}
\item 製品実体によって分割
    \begin{itemize}
    \item 例えば外部関数(黒箱視点)又はシステム部品(白箱視点)
    \item<uncover@2-|alert@2>
    チェックリストと同じ
    \end{itemize}
\item 性質、関係、論理的な条件によって分割、更に2つに分かれている
    \begin{itemize}
    \item 論理変数を論理演算子に結ばれた論理述語による \note[item]{述語:じゅつご}
    \item 数字変数を比較演算子によって比較する
    \item<uncover@3-|alert@3>
    論理述語や比較演算によって入力区間を分割
    \item<uncover@4-|alert@4>
    決定木を使う
    \end{itemize}
\item 1と2の組み合わせ 
\end{enumerate}
\end{frame}
%%%%%%%%%%%%%%%%%%%%%%%%%%%%%%%%%%%%%%%%%%%%%%%%%%%%%%%%%%%%%%%%%%%%%%%%
\begin{frame}[fragile]{決定木によって分割}
\framesubtitle{Testing Decisions based on Decision Tree}
\transfade<2->[duration=0.5]

\begin{columns}
\begin{column}{0.5\textwidth}

\tikzstyle{level 1}=[level distance=1.5cm, sibling distance=2.5cm]
\tikzstyle{level 2}=[level distance=1cm, sibling distance=2cm]
\tikzstyle{level 3}=[level distance=1cm, sibling distance=1.5cm]
\tikzstyle{bag} = [rectangle, text width=6em, text centered, draw]
\tikzstyle{decision} = [rectangle, minimum height=8pt, minimum width=8pt, fill, inner sep=0pt]
\tikzstyle{choice} = [circle, minimum width=8pt, fill, inner sep=0pt]
\tikzstyle{end} = [regular polygon, regular polygon sides=3, minimum width=8pt, fill, inner sep=0pt,shape border rotate=90]

\begin{tikzpicture}[grow=right,child anchor=west]
\tiny
\node[bag]{\only<1|handout:0|trans:0|article:0>{最初決定}\only<2|handout:1|trans:1|article:1>{決定$P$}}
child {
    node[decision]{}
    child {
        node[bag]{\only<1|handout:0|trans:0|article:0>{中間決定1}\only<2|handout:1|trans:1|article:1>{決定$Q$}}
            child{
                node[choice]{}
                child {
                    node[bag]{\only<1|handout:0|trans:0|article:0>{結果1}\only<2|handout:1|trans:1|article:1>{$P \wedge Q$}}
                    child {
                        node[end]{}
                    }
                }
                child {
                    node[bag]{\only<1|handout:0|trans:0|article:0>{結果2}\only<2|handout:1|trans:1|article:1>{$P \wedge \neg Q$}}
                    child {
                        node[end]{}
                    }
                }
            }
    }
    child {
        node[bag]{\only<1|handout:0|trans:0|article:0>{中間決定2}\only<2|handout:1|trans:1|article:1>{決定$Q$}}
            child{
                node[choice]{}
                child {
                    node[bag]{\only<1|handout:0|trans:0|article:0>{結果3}\only<2|handout:1|trans:1|article:1>{$\neg P \wedge Q$}}
                    child {
                        node[end]{}
                    }
                }
                child {
                    node[bag]{\only<1|handout:0|trans:0|article:0>{結果4}\only<2|handout:1|trans:1|article:1>{$\neg P \wedge \neg Q$}}
                    child {
                        node[end]{}
                    }
                }
            }
        }
    };
\end{tikzpicture}

\only<1|handout:0|trans:0|article:0>{一貫性がない決定木}
\only<2|handout:1|trans:1|article:1>{一貫性を持つ決定木}

\uncover<2>{決定木の上、一つの結果は一つの分割方法となって、パスを沿って結果を成り立たせる入力の範囲を求める}
\end{column}

\begin{column}{0.5\textwidth}
\begin{lstlisting}[language=java,escapechar=@]
if(@\only<1|handout:0|trans:0|article:0>{@最初決定@}\only<2|handout:1|trans:1|article:1>{@P@}@){
    if(@\only<1|handout:0|trans:0|article:0>{@中間決定1@}\only<2|handout:1|trans:1|article:1>{@Q@}@){
        @\only<1|handout:0|trans:0|article:0>{@結果1@}\only<2|handout:1|trans:1|article:1>{@$P \wedge Q$@}@;
    }else{
        @\only<1|handout:0|trans:0|article:0>{@結果2@}\only<2|handout:1|trans:1|article:1>{@$P \wedge \neg Q$@}@;
    }
}else{
    if(@\only<1|handout:0|trans:0|article:0>{@中間決定2@}\only<2|handout:1|trans:1|article:1>{@Q@}@){
        @\only<1|handout:0|trans:0|article:0>{@結果3@}\only<2|handout:1|trans:1|article:1>{@$\neg P \wedge Q$@}@;
    }else{
        @\only<1|handout:0|trans:0|article:0>{@結果4@}\only<2|handout:1|trans:1|article:1>{@$\neg P \wedge \neg Q$@}@;
    }
}
\end{lstlisting}
\end{column}
\end{columns}
\end{frame}
%%%%%%%%%%%%%%%%%%%%%%%%%%%%%%%%%%%%%%%%%%%%%%%%%%%%%%%%%%%%%%%%%%%%%%%%
\section{8.3 Musa氏の操作分布で使用ベース統計的テスト}
\subsection{8.3.1 使用ベース統計的テストの場合}
\begin{frame}[shrink=13]{使用ベース統計的テストが適用する場合}
\framesubtitle{The Cases for Usage-based Statistical Testing}
\transwipe<2->

\begin{columns}
\begin{column}{0.4\textwidth}
\begin{tabular}{lrl}
\hline
  操作   & \uncover<2->{頻度(\%)} & \\
\hline
スブドメイン1 &                      & \\
-操作1   & \uncover<2->{ 1\% }  & \\
-操作2   & \uncover<2->{72\% }  & \uncover<3->{$\checkmark$} \\
-操作3   & \uncover<2->{ 2\% }  & \\
スブドメイン2 &                      & \\
-操作4   & \uncover<2->{13\% }  & \uncover<3->{$\checkmark$} \\
-操作5   & \uncover<2->{ 2\% }  & \\
スブドメイン3 &                      & \\
-操作6   & \uncover<2->{ 3\% }  & \\
-操作7   & \uncover<2->{ 7\% }  & \uncover<3->{$\checkmark$} \\
\hline
\end{tabular}
\end{column}
\begin{column}{0.6\textwidth}
\only<4->{
\begin{enumerate}
\item 使用シナリオ、パターン、関連する使用頻度を、目標消費者やユーザーから収集
\item 収集した情報を分析し、操作分布(OP, Operation Profile)に整理
\item 操作分布によってテストを行い
\item テストの結果を分析し、製品の信頼性を評価し、テストのフィードバックやソフトウェア開発プロセスに役立つ
    \begin{itemize}
    \item<uncover@5-> 製品の信頼性を評価する方法は22章に紹介する、他のテストに関係する活動は7章に紹介した
    \end{itemize}
\end{enumerate}
}
\end{column}
\end{columns}
\end{frame}
%%%%%%%%%%%%%%%%%%%%%%%%%%%%%%%%%%%%%%%%%%%%%%%%%%%%%%%%%%%%%%%%%%%%%%%%
\begin{frame}{操作分布を開発プロセスに導入時期}
\framesubtitle{The Period of Development Process that Involves Operational Profiles}
\transfade<2->[duration=0.5]
\mode<article>{\renewcommand{\imageheight}{0.2\textheight}}
\mode<presentation>{\renewcommand{\imageheight}{0.7\textheight}}
\pgfdeclareimage[height=\imageheight]{vmodel}{figure/V-model.JPG}
\pgfdeclareimage[height=\imageheight]{vmodel-common}{figure/V-model-common.png}
\pgfdeclareimage[height=\imageheight]{vmodel-op}{figure/V-model-OP.png}
\begin{figure}
\begin{center}
\begin{overprint}
\onslide<1|handout:0|trans:0|article:0> \begin{center}\pgfuseimage{vmodel}\end{center}
\onslide<2|handout:0|trans:0|article:0> \begin{center}\pgfuseimage{vmodel-common}\end{center}
\onslide<3|handout:1|trans:1|article:1> \begin{center}\pgfuseimage{vmodel-op}\end{center}
\end{overprint}
\caption{操作分布を開発プロセスに導入する時期}
\end{center}
\end{figure}
\end{frame}
%%%%%%%%%%%%%%%%%%%%%%%%%%%%%%%%%%%%%%%%%%%%%%%%%%%%%%%%%%%%%%%%%%%%%%%%
\subsection{8.3.2 Musa操作分布:基本的な考え方(Musa OP:Basic Ideas)}
\begin{frame}[shrink=10]{Musa操作分布:基本的な考え方(定義)}
\framesubtitle{Musa OP:Basic Ideas(Defination)}
\begin{definationfc}[操作分布]
操作分布とは、一連の操作とそれらの出現に関連する確率の配列 

An operational profile is a list of disjoint set of operations and their associated probabilities of occurrence

\flushright{---\cite{musa1993operational}}
\end{definationfc}
\bibliographystyle{IEEEtranN}
\bibliography{chapter8}
\end{frame}

%%%%%%%%%%%%%%%%%%%%%%%%%%%%%%%%%%%%%%%%%%%%%%%%%%%%%%%%%%%%%%%%%%%%%%%%

\begin{frame}[shrink=10]{Musa操作分布:基本的な考え方(例)}
\framesubtitle{Musa OP:Basic Ideas(Example)}
\mode<article>{\renewcommand{\imageheight}{0.2\textheight}}
\mode<presentation>{\renewcommand{\imageheight}{0.6\textheight}}
\pgfdeclareimage[height=\imageheight]{OPfiles}{figure/OPfiles.png}
\pgfdeclareimage[height=\imageheight]{OPgraph}{figure/OPgraph.png}
\begin{columns}

\begin{column}{0.4\textwidth}
\begin{figure}
\begin{center}
\pgfuseimage{OPfiles}
\caption{SMU/SEASの各ファイル型の使用頻度とその確率}
\end{center}
\end{figure}
\end{column}
\begin{column}{0.6\textwidth}
\begin{figure}
\begin{center}
\pgfuseimage{OPgraph}
\caption{SMU/SEASの各ファイル型の使用確率}
\end{center}
\end{figure}
\end{column}
\end{columns}

\note[item]{使用確率によってsortする必要は限られていない}
\note[item]{使用確率の分布が均一(きんいつ)とは限られていない}
\note[item]{均一(きんいつ)とは限られていないから、閾値を決めてフィルタリングすることが可能}
\note[item]{具体的に、操作分布はチェックリストでもあるし、分割でもあり。}
\end{frame}

%%%%%%%%%%%%%%%%%%%%%%%%%%%%%%%%%%%%%%%%%%%%%%%%%%%%%%%%%%%%%%%%%%%%%%%%
\subsection{8.3.3 操作分布を用いて統計的などのテスト等に使う}
\begin{frame}[shrink=10]{操作分布を用いて\alert{統計的}なテストに使う}
\framesubtitle{Using OPs for Statistical Testing }
\begin{center}
\begin{tabular}{lrr}
\hline
  操作   & \uncover<2->{頻度(\%)} & \uncover<3->{テスト数(合計30)}\\
\hline
スブドメイン1 &                      & \\
-操作1   & \uncover<2->{ 1\% }  & \uncover<3->{0}  \\
-操作2   & \uncover<2->{72\% }  & \uncover<3->{23} \\
-操作3   & \uncover<2->{ 2\% }  & \uncover<3->{0} \\
スブドメイン2 &                      & \\
-操作4   & \uncover<2->{13\% }  & \uncover<3->{4} \\
-操作5   & \uncover<2->{ 2\% }  & \uncover<3->{0} \\
スブドメイン3 &                      & \\
-操作6   & \uncover<2->{ 3\% }  & \uncover<3->{1} \\
-操作7   & \uncover<2->{ 7\% }  & \uncover<3->{2} \\
\hline
\end{tabular}
\end{center}
\end{frame}
%%%%%%%%%%%%%%%%%%%%%%%%%%%%%%%%%%%%%%%%%%%%%%%%%%%%%%%%%%%%%%%%%%%%%%%%
\begin{frame}{操作分布を用いて\alert{進歩的}なテストに使う}
\framesubtitle{Using OPs for Progressive Testing}
\transfade<2->[duration=0.1]
\animate<2-7>
\begin{center}
\mode<article>{\def\tikzy{0.04cm}\def\tikzx{0.6cm}}
\mode<presentation>{\def\tikzy{0.07cm}\def\tikzx{1cm}}
\begin{tikzpicture}[x=\tikzx,y=\tikzy]
\draw[ycomb, color=gray, line width=0.5cm] plot file {data-prog};
\draw[->] (-0.5,0) -- (8,0) node[right] {操作};
\only<1>{\draw[-] (-0.5,75) -- (8,75) node[right] {};}
\only<2>{\draw[-] (-0.5,50) -- (8,50) node[right] {};}
\only<3>{\draw[-] (-0.5,15) -- (8,15) node[right] {};}
\only<4>{\draw[-] (-0.5,10) -- (8,10) node[right] {};}
\only<5>{\draw[-] (-0.5,6) -- (8,6) node[right] {};}
\only<6>{\draw[-] (-0.5,3) -- (8,3) node[right] {};}
\only<7>{\draw[-] (-0.5,2) -- (8,2) node[right] {};}
\only<8>{\draw[-] (-0.5,1) -- (8,1) node[right] {};}
\draw[->] (0,-0.5) -- (0,75) node[above] {テストケース数};
\end{tikzpicture}
\end{center}
\end{frame}
%%%%%%%%%%%%%%%%%%%%%%%%%%%%%%%%%%%%%%%%%%%%%%%%%%%%%%%%%%%%%%%%%%%%%%%%
\begin{frame}{他に操作分布の使う目的}
\framesubtitle{OPs Used in Other Situations for Various Purpose}
\begin{itemize}
\item 進歩的テストによって、ソフトウェアの信頼性目標に達成するまで予測
    \note[item]{信頼性目標の予測を22章に紹介する}
\item 開発効率の向上
    \note[item]{あんまり使われていない部分のテストを防いだ}
    \begin{itemize}
    \item AT\&Tで行われた実験のデータによると、56\%のシステム・テスト・コストと11.5\%の全体のコストを削減(さくげん)した
    \end{itemize}
\item よく使われる機能を特定し、それに着目して新しい機能や製品を開発
    \begin{itemize}
    \item あまり使われていない機能は後で
    \item 螺旋(spiral)開発やプロトタイプ開発などのプロセスに使える
    \end{itemize}
\item 顧客とより良い交流とより深い関係
    \begin{itemize}
    \item 顧客から製品の品質や機能に対する意見を把握できる
    \item もっと細かい要求分析と仕様が可能になる
    \item 顧客トレーニングを実行すべき項目を特定できる
    \end{itemize}
\item 投資の高い収益率
    \begin{itemize}
    \item 線形に近いコストの成長率
    \end{itemize}
    \note[item]{AT\&Tのデータによると、普通の製品に対すOP開発コストは1人月。}
    \note[item]{普通の定義は10万行、10人の開発者で一年半}
\end{itemize}
\end{frame}
\end{document}
